\documentclass[10pt]{article}
\usepackage[left=0.65in,top=0.7in,right=.65in,bottom=0.7in]{geometry} % Document margins
\usepackage{xhfill}
\usepackage{stackengine}
\usepackage{float}
\usepackage{verbatimbox}
\usepackage{enumitem}
\setlist[itemize]{noitemsep, topsep=0pt, leftmargin=.5in}
\setlist[enumerate]{topsep=0pt, leftmargin=.5in}
%\usepackage{helvet}
%\renewcommand{\familydefault}{\sfdefault}


\begin{document}
	
	\renewcommand\labelitemi{\tiny$\bullet$}
	
	\begin{center}
		\textbf{\Large Sheila M. Gaynor}\\
		\vspace{0.1cm}
		\begin{table}[H]
			\centering
			\begin{tabular}{@{}p{6cm}@{}@{}p{6cm}@{}}				
				Department of Biostatistics  & Cell: (919) 656-8433                \\
				655 Huntington Avenue  &  Email: sheilagaynor@hsph.harvard.edu               \\
				 Building 2, Room 435     & Website: https://sheilagaynor.com   \\
				Boston, MA 02115             & GitHub: github.com/sheilagaynor   \\     
			\end{tabular}
		\end{table}
	\end{center}

\vspace{0.3cm}
{ \bf EDUCATION} \hrulefill \\

\textbf{Harvard University} \hfill 08/2013-05/2018\\
\indent Ph.D. in Biostatistics, May 2018  \hfill Cambridge, MA
\begin{itemize}
	\item Dissertation Title: “Statistical Methods for Integratively Characterizing Genetic and Genomic Data”
	\item Advisors:  Dr. John Quackenbush and Dr. Xihong Lin
\end{itemize}

\indent A.M. in Biostatistics, May 2015\\

\textbf{University of North Carolina at Chapel Hill} \hfill 08/2009-05/2013\\
\indent B.S.P.H in Biostatistics and  B.A. in Mathematics, May 2013   \hfill Chapel Hill, NC
\begin{itemize}
	\item Honors Thesis Title: “Identification of Biologically Relevant Subtypes via Preweighted Sparse Clustering”
	\item With highest honors and highest distinction\\
\end{itemize}

\vspace{0.3cm}


{\bf ACADEMIC APPOINTMENT} \hrulefill \\

\textbf{Harvard T.H. Chan School of Public Health} \hfill 06/2018-Present\\
\indent Postdoctoral Fellow, Department of Biostatistics \hfill Boston, MA
\begin{itemize}
	\item Fellowship with Dr. Xihong Lin
	\item Develop statistical and computational methods to analyze whole genome sequencing, genomic, and clinical data
	\item Collaborate on large-scale studies including the NHLBI Trans-Omics for Precision Medicine (TOPMed) Program
	\item Focus on rare variant, fine mapping, and functional annotation methods and applications \\
\end{itemize}

\vspace{0.3cm}

{ \bf EXPERIENCE} \hrulefill \\

\textbf{Center for Translational Pain Medicine, Duke University} \hfill 2017-Present\\
\indent Statistical Consultant  \hfill Durham, NC\\
\indent \textit{Analysis of patients with pain phenotyping and development of clinically-implemented algorithm of pain subtyping \\}

\textbf{Behavioral Science Research, Boston University} \hfill 2015-2017\\
\indent Visiting Researcher \hfill Boston, MA\\
\indent \textit{Modeled a study of smokers unmotivated to quit using latent class analysis  \\}

\textbf{Neurobiology of Fear Laboratory, McLean Hospital} \hfill 2015-2017\\
\indent Visiting Researcher \hfill Belmont, MA\\
\indent \textit{Analyzed a study of post-traumatic stress disorder using network and mediation analysis \\}

\textbf{Department of Biomedical Informatics, Harvard University} \hfill 2014\\
\indent Rotation Student \hfill Boston, MA\\
\indent \textit{Implemented bioinformatics tools to call copy number variants in The Cancer Genome Atlas (TCGA)  \\}

\textbf{Department of Biostatistics, University of North Carolina} \hfill 2011-2013\\
\indent Honors Undergraduate Researcher \hfill Chapel Hill, NC\\
\indent \textit{Developed sparse clustering method for data with high-variance features or an outcome in supervised settings \\}

\textbf{Department of Biostatistics, University of North Carolina} \hfill 2012\\
\indent Summer Undergraduate Research Fellow \hfill Chapel Hill, NC\\
\indent \textit{Analyzed the association between hormonal contraceptive use and pain conditions \\}

\textbf{Outcome Research Office, Washington University in St. Louis} \hfill 2010-2011\\
\indent Statistical Consultant and Intern \hfill St. Louis, MO\\
\indent \textit{Investigated comorbidity scoring methods and prognostic models for cancer survival in hospital biobanks \\}

\textbf{Department of Biostatistics, Washington University in St. Louis} \hfill 2010\\
\indent Summer Institute for Training in Biostatistics Trainee \hfill St. Louis, MO\\
\indent \textit{Learned and applied principles of applied biostatistics \\}

\vspace{0.25cm}


{\bf FELLOWSHIPS AND GRANTS} \hrulefill \\

\indent \textbf{Awarded} \\

\textbf{NIH F31 Kirschstein Predoctoral Individual National Research Service Award} \hfill 09/17-05/18\\
\indent NIH/NHLBI HL138832
\begin{itemize}
	\item \textit{Integrative analysis of lung disease genotypes and gene expression}
	\item Goal: To develop statistical methods and open source tools for jointly analyzing genetic and genomic data to understand the genetic basis of lung diseases\\
\end{itemize}

\textbf{NSF Graduate Research Fellowship} \hfill 08/13-05/18\\
\indent NSF DGE1144152, DGE1745303
\begin{itemize}
	\item \textit{Graduate Research Fellowship Program in Mathematical Sciences - Biostatistics}
	\item Goal: To obtain training in biostatistical theory, methods, and computation and support doctoral research\\
\end{itemize}


\indent \textbf{Under Review}  \\

\textbf{K99/R00 Pathway to Independence Award} \hfill Submitted 06/19\\
\indent NIH/NHLBI
\begin{itemize}
	\item \textit{Statistical methods to integrate rich functional and phenotypic data in whole genome sequencing analyses}
	\item Goal: To develop statistical and computational methods for massive whole genome sequencing data and transition to independent research \\
\end{itemize}

\vspace{0.25cm}

{ \bf HONORS AND AWARDS} \hrulefill \\
 \setlength\intextsep{0mm}
\begin{center}
	\begin{table}[H]
		\centering
		\begin{tabular}{@{}p{1.3cm}@{}@{}p{16.2cm}@{}}
			2017  & Summer Institute in Statistical Genetics Scholarship, University of Washington Department of Biostatistics  \\ 
			2017  & Program in Quantitative Genomics Travel Award, Harvard T.H. Chan School of Public Health  \\ 
			2016  & XSEDE Computation Allocation, NSF            \\ 
			2016  & Certificate of Distinction in Teaching, Harvard University Department of Biostatistics             \\ 
			2013 & T32 NIH HIV/AIDS Training Grant Fellowship, NIAID \\
			2013  & Delta Omega Undergraduate Award, Delta Omega Public Health Honors Society       \\ 
			2013  & Carolina Research Scholar, UNC Office of Undergraduate Research   \\ 
			2013  & Buckley Public Service Scholar, UNC Carolina Center for Public Service    \\ 
			2012  &Phi Beta Kappa, University of North Carolina \\
			2012  & Summer Undergraduate Research Fellowship, University of North Carolina  
		\end{tabular}
	\end{table}
\end{center}

%\vspace{0.2cm}

{ \bf PUBLICATIONS} \hrulefill \\

\indent \textbf{Peer-Reviewed} \\
\begin{enumerate}
\item \textbf{Gaynor, S.M.}, Schwartz, J., and Lin, X. (2019). Mediation analysis for common binary outcomes. Statistics in Medicine, 38(4), 512-529.
\item \textbf{Gaynor, S.M.}*, Sun, R.*, Lin, X., and Quackenbush, J. (2019). Identification of differentially expressed gene sets using the Generalized Berk–Jones statistic. Bioinformatics.
\item Borrelli, B., \textbf{Gaynor, S.}, Tooley, E., Armitage, C. J., Wearden, A., \& Bartlett, Y.K. (2018). Identification of three different types of smokers who are not motivated to quit: Results from a latent class analysis. Health Psychology, 37(2), 179.
\item \textbf{Gaynor, S.}, and Bair, E. (2017). Identification of relevant subtypes via preweighted sparse clustering. Computational Statistics and Data Analysis, 116, 139-154.
\item Bair, E., \textbf{Gaynor, S.}, Slade, G.D., Ohrbach, R., Fillingim, R.B., Greenspan, J.D., Dubner, R., Smith, S.B., Diatchenko, L., and Maixner, W. (2016). Identification of clusters of individuals relevant to temporomandibular disorders and other chronic pain conditions: the OPPERA study. Pain, 157(6), 1266.
\item Kallogjeri, D., \textbf{Gaynor, S.M.}, Piccirillo, M.L., Jean, R.A., Spitznagel Jr, E.L., \& Piccirillo, J.F. (2014). Comparison of comorbidity collection methods. Journal of the American College of Surgeons, 219(2), 245-255.\\
\end{enumerate}

\vspace{0.2cm}

\indent \textbf{Submitted and Preprints} \\

\begin{enumerate}
\item \textbf{Gaynor, S.M.}, Fagny, M., Lin, X., Platig, J., and Quackenbush, J. Connectivity of variants in eQTL networks dictates reproducibility and functionality. bioRxiv 515551 [Preprint].  
\item \textbf{Gaynor, S.M.}, Lin, X., and Quackenbush, J. Spectral clustering in regression-based biological networks. bioRxiv 651950 [Preprint].
\item \textbf{Gaynor, S.M.}, Fillingim, R.B., Zolnoun, D.A., Slade, G.D., Ohrbach, R., Greenspan, J.D., Maixner, W.,  and Bair, E. Association between craniofacial pain and hormonal contraceptive use: The OPPERA study.
\item \textbf{Gaynor, S.M.}, Bortsov, A., Maixner, W.,  and Smith, S.B. Pragmatic patient profile clustering identifies diagnostically and prognostically informative subgroups.
\item McCaw, Z.R., \textbf{Gaynor, S.M.}, Sun, R.,  and Lin, X. Cross-tissue eQTL Calling via Surrogate Expression Analysis.
\item Sun, R.*, Xu, M.*, Li, X., \textbf{Gaynor, S.M.}, Zhou, H., Bosse, Y., Lam, S., Tsao, M., Tardon, A., Chen, C., Doherty, J., Goodman, G., Egil Bojesen, S., Teresa, M.T., Johansson, M., Field, J.K., Bickeboller, H, Wichmann, H., Risch, A., Rennert, G., Arnold, S., Wu, X., Melander, O., Brunnstrom, H.,
Marchand, L.L., Zong, X., Liu, G., Andrew, A., Duell, E., Kiemeney, L.A., Shen, H., Haugen, A.,
Johansson, M., Grankvist, K., Caporaso, N., Woll, P., Teare, M.D., Scelo, G., Hong, Y., Yuan, J.,
Lazarus, P., Schabath, M.B., Aldrich, M.C., Albanes, D., Brennan, P., Barbie, D., Mak, R., Hung,
R.J., Amos, C.I., Christiani, D.C and Lin, X. Identification of inflammation and immune-related risk variants associated with squamous cell lung cancer.
\item Li, X.*, Li, Z.*, Zhou, H., \textbf{Gaynor, S.M.}, Liu, Y., Chen, H., Sun, R., Dey, R., Arnett, D.K., Aslibekyan, S., Ballantyne, C.M., Bielak, L.F., Blangero, J., Boerwinkle, E., Bowden, D.W., Broome, J.G., Conomos, M.P., Correa, A., Curran, J.E., Cupples, L.A., Freedman, B.I.,  Guo, X., Kardia, S.L.R., Kathiresan, S., Khan, A.T.,  Kooperberg, C.L., Irvin, M.R., Laurie. C.C., Manichaikul, A.W., Mahaney, M.C., Mathias, R.A., Morrison, A.C., Martin, L.W., McGarvey, S.T., Mitchell, B.D., Montasser, M.E.,  Moore, J.,  O’Connell, J.R., Palmer, N.D., Pampana, A.,  Peralta, J.M.,  Peyser, P.A.,  Psaty, B.M., Vasan, R.S., Redline, S., Rice, K.M., Rich, S.S., Smith, J.A.,  Tsai, M., Tiwari, H.K., Wang, F.F., Weeks, D.E., Weng, Z., Wilson, J.G., Yanek, L.R., NHLBI Trans-Omics for Precision Medicine (TOPMed) Consortium, TOPMed Lipids Working Group, Neale, B.M., Sunyaev, S.R., Abecasis, G.R.,  Rotter, J.I.,  Willer, C.J., Peloso G.M., Natarajan, P., and Lin, X. Dynamic incorporation of multiple in-silico functional annotations empowers rare variant association analysis of large whole genome sequencing studies at scale.
\item Raffield, L.M.,  Iyengar, A.K.,  Wang, B.,  \textbf{Gaynor, S.M.}, Spracklen, C.N., Kowalski, M.H., Salimi, S., Polfus, L.M., Benjamin, E.J., Bis, J.C., Bowler, R.,Cade, B.E., Comellas. A.P.,  Correa, A.,  Durda, P.,  Gogarten, S.,  Jain, D., Kral, B.G., Lange, L.A.,  Larson, M.G.,  Laurie, C., Lee, J., Lewis, J.P., Mitchell, B.,  Pankratz, N.,  Rich, S.S., Rotter, J.I., Ryan, K., Tracy, R.P., Yanek, L.R.,  Zhao, L.P., Lin, X., Li, Y., Dupuis, J., Reiner, A.P., Mohlke, K.L., Auer, P.L., TOPMed Inflammation Working Group, and NHLBI Trans-Omics for Precision Medicine (TOPMed) Consortium. Allelic heterogeneity at CRP locus identified by whole-genome sequencing in multi-ancestry cohorts.\\
\end{enumerate}

\indent \indent  * Indicates equal contribution as first authors\\

\pagebreak

\indent \textbf{Book Chapters} \\
\begin{enumerate}
	\item Klengel, T., Lebois, L.A.M., \textbf{Gaynor, S.M.}, and Guffanti, G. (2018). Genetics and Gene-Environment Interaction. In Stoddard Jr, F.J.,
Benedek, D.M., Milad, M.R., and Ursano, R. J. (Eds.). Trauma-and stressor-related disorders. Oxford University Press. (198-210).\\
\end{enumerate}

\vspace{0.cm}

{\bf  EDUCATIONAL CONTRIBUTIONS} \hrulefill \\

\indent \textbf{Teaching} 
\begin{center}
	\begin{table}[H]
		\centering
		\begin{tabular}{@{}p{2cm}@{}@{}p{14.9cm}@{}}
			2017  & Teaching Assistant, ID 201: Principles of Biostatistics and Epidemiology, Harvard University Department of Biostatistics                 \\ 
			2017  & Guest Lecturer, BST 254: Effective Grant \& Research Proposal Writing for Biostatistics Research, Harvard University Department of Biostatistics                 \\ 
			2016  & Director of Independent Study in Biostatistics, Harvard University Department of Biostatistics                 \\ 
			2015, 2016  & Biostatistics and Epidemiology Program Tutor, Harvard University Commonwealth Fund Fellowship in Minority Health Policy                 \\ 
			2015  & Founding Head Teaching Assistant, ID 201: Principles of Biostatistics and Epidemiology, Harvard University Department of Biostatistics                 \\ 
			2015  & Guest Lecturer, ID 201: Principles of Biostatistics and Epidemiology, Harvard University Department of Biostatistics                 \\ 
			2014  & Teaching Assistant, BIO 200: Principles of Biostatistics, Harvard University Department of Biostatistics                
		\end{tabular}
	\end{table}
\end{center}

\textbf{Advising and Mentorship} 
\setlength\intextsep{0mm}
\begin{center}
	\begin{table}[H]
		\centering
		\begin{tabular}{@{}p{2cm}@{}@{}p{14.9cm}@{}}
			2019  & Claire Tseng, \textit{Undergraduate Independent Study, Biostatistics, Harvard University}  \\ 
			2018-2019  & Wenying Deng, \textit{Master’s Research Assistantship, Biostatistics, Harvard University}   \\ 
			2018  & Mengting Li,  \textit{Master’s Thesis Committee, Biostatistics, Harvard University }  \\ 
			2017  &  \textit{Graduate Mentor, Harvard Summer Program in Biostatistics \& Computational Biology   } 
		\end{tabular}
	\end{table}
\end{center}



{ \bf PRESENTATIONS} \hrulefill \\


\textbf{Oral Presentations }  \setlength\intextsep{0mm}
\begin{center}
	\begin{table}[H]
		\centering
		\begin{tabular}{@{}p{2cm}@{}@{}p{14.9cm}@{}}
			06/2019  &  Dynamic incorporation of functional annotations in rare variant analysis. \textit{Framingham Heart Study FOCuS Seminar Series. }          \\ 
			04/2019  &  Identification of rare variant sets associated with lung function via set-based analyses incorporating in-silico functional annotations. \textit{NHLBI TOPMed In-Person Meeting. }          \\ 
			07/2018  & Degree centrality of SNPs in eQTL networks. \textit{Joint Statistical Meetings. }              \\ 
			04/2017  & Reframing eQTL Networks to account for intermediate analyses. \textit{Biostatistics Student Seminar. }\\
			03/2017  & Error quantification in biologically relevant eQTL network metrics. \textit{ENAR Spring Meeting.} \\
			03/2017  & Mediation analysis of pathways to PTSD diagnosis. \textit{Massachusetts General Hospital Trauma Genomics Group.} \\
			02/2017  & Causal mediation analysis for genomic data. \textit{Broad Institute Statistical Genetics Seminar.}  \\
			10/2016  & Graduate Research Fellowships. \textit{Biostatistics Student Seminar.} \\
			07/2016  & Genomic analysis with common binary outcomes via mediation. \textit{Joint Statistical Meetings.}\\
			03/2016  & Mediation methods for case-control settings with applications to genomics.\textit{ ENAR Spring Meeting.} \\
			10/2015  & Integrating epigenetic and genomic analyses via mediation analysis. \textit{Harvard Medical School Epigenetics Symposium. }\\
			08/2015  & A mediation-based integrative genomic analysis of lung cancer. \textit{Biostatistics Student Seminar. }\\
			07/2015  & Mediation-based integrative genomic analysis. \textit{Joint Statistical Meetings.} \\
			03/2015  & Mediation-based integrative genomic analysis. \textit{ENAR Spring Meeting.}\\
			03/2015  & Preweighted sparse clustering with applications to temporomandibular disorder. \textit{International Association for Dental Research Epi-Forum.} \\
			03/2014  & Identification of biologically relevant subtypes via preweighted sparse clustering. \textit{ENAR Spring Meeting.}
		\end{tabular}
	\end{table}
\end{center}


\textbf{Poster Presentations} \setlength\intextsep{0mm}
\begin{center}
	\begin{table}[H]
		\centering
		\begin{tabular}{@{}p{2cm}@{}@{}p{14.9cm}@{}}
			07/2019  & Rare variant association analysis for studies with rich phenotyping subject to missingness. \textit{American Society for Human Genetics Annual Meeting.}  \\ 
			07/2019  & Fine mapping causal variants with functional annotations. \textit{Joint Statistical Meetings. Oral poster.}  \\ 
			03/2018  & Assessing the Effective Degree of SNPs in eQTL Networks. \textit{ENAR Spring Meeting. Oral poster.}  \\ 
			10/2016  & Mediation methods applied to post-traumatic stress disorder to identify genomic effects. \textit{American Society for Human Genetics Annual Meeting. } \\ 
			04/2015  & A mediation-based integrative genomic analysis of lung cancer. \textit{Harvard Graduate Women in Science and Engineering Symposium.}    \\ 
			04/2013  & The association between oral contraceptive use and painful conditions. \textit{UNC Celebration of Undergraduate Research. }           \\ 
			10/2012  & OPPERA Study Identifies an Association Between the Use of Hormonal Contraceptives and Orofacial Pain and Headaches.\textit{ International Pelvic Pain Society Conference. Prepared poster, First place in poster competition. }      \\ 
			04/2012  & Identification of clinically relevant disease subtypes using supervised sparse clustering.  \textit{ENAR Spring Meeting. }
		\end{tabular}
	\end{table}
\end{center}

	
{ \bf PROFESSIONAL SERVICE} \hrulefill \\
\setlength\intextsep{0mm}
\begin{center}
	\begin{table}[H]
		\centering
		\begin{tabular}{@{}p{2cm}@{}@{}p{14.9cm}@{}}
			2018-2019  & Organizer, Harvard Biostatistics Postdoctoral Mental Health Study         \\ 
			2017-2018  & Member, Harvard Biostatistics Colloquium Committee          \\ 
			2016-2018  & Co-organizer, Harvard Biostatistics-Biomedical Informatics Big Data Seminar             \\ 
			2015-2017  & Member, Harvard Biostatistics Student Advising Committee    \\ 
			2016  & Graduate Participant, International High Performance Computing Summer School   \\ 
			2015-2016  & Organizer \& Coordinator, Harvard Big Data Seminar    \\ 
			2015  & Chair, ENAR Session on Graphical Modeling  \\ 
			2014  & Judge, Harvard School of Public Health Poster Day 
		\end{tabular}
	\end{table}
\end{center}


{ \bf TECHNICAL SKILLS} \hrulefill \\
\setlength\intextsep{0mm}
\begin{center}
	\begin{table}[H]
		\centering
		\begin{tabular}{@{}p{3.9cm}@{}@{}p{13cm}@{}}
			Statistical Software  &  R, Experience in: SAS, STATA, Python       \\ 
			Genetics Software  & PLINK, GCTA   \\ 
			Other Software  & Linux computing, GitHub, \LaTeX, Microsoft Office, Cytoscape
		\end{tabular}
	\end{table}
\end{center}


{ \bf REFERENCES} \hrulefill \\

\indent \indent Xihong Lin, Professor\\
\indent \indent Department of Biostatistics and Department of Statistics\\
\indent \indent Harvard T.H. Chan School of Public Health\\
\indent \indent xlin@hsph.harvard.edu, (617) 432-2914
\vspace{0.35cm}

\indent \indent John Quackenbush, Henry Pickering Walcott Professor of Computational Biology and Bioinformatics. Chair\\
\indent \indent Department of Biostatistics\\
\indent \indent Harvard T.H. Chan School of Public Health\\
\indent \indent johnq@hsph.harvard.edu, (617) 432-9028
\vspace{0.35cm}

\indent \indent William Maixner, Joannes H. Karis M.D. Professor of Anesthesiology\\
\indent \indent Department of Anesthesiology\\
\indent \indent Duke University School of Medicine \\
\indent \indent william.maixner@duke.edu, (919) 681-9933
\vspace{0.35cm}

\indent \indent Paige Williams, Senior Lecturer and Director of Graduate Studies\\
\indent \indent Department of Biostatistics and Department of Epidemiology\\
\indent \indent Harvard T.H. Chan School of Public Health\\
\indent \indent paige@hsph.harvard.edu, (617) 432-3872\\


\end{document}